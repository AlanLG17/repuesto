\documentclass[11pt]{scrartcl}
\usepackage[sexy]{evan}
\usepackage{amsthm}
\usepackage[helvetica]{quotchap} 

\title{AaAaaAaAAaaAaAaaa}
\author{AlanLG}
\date{10 de Marzo del 2024}

\begin{document}

\maketitle

\epigraph{He lamentado profundamente no haber avanzado al menos lo suficiente como para comprender algo de los grandes principios fundamentales de las matemáticas, pues los hombres que los dominan parecen poseer un sexto sentido}
{\emph{Charles Darwin.}}


\begin{problem}[\href{https://www.proyectomuro.com/OMMEB/2017/}{OMMEB 2017 N2}]
    
   
    En el hexágono regular, intersectamos dos de las diagonales . Encuentra la razón entre el área roja y el área azul
    
\begin{center}
    \begin{asy}
         /* Geogebra to Asymptote conversion, documentation at artofproblemsolving.com/Wiki go to User:Azjps/geogebra */
import graph; size(4.5cm); 
real labelscalefactor = 0.5; /* changes label-to-point distance */
pen dps = linewidth(0.7) + fontsize(10); defaultpen(dps); /* default pen style */ 
pen dotstyle = black; /* point style */ 
real xmin = -4.84, xmax = 14.72, ymin = -3.32, ymax = 8.56;  /* image dimensions */


draw((0,0)--(4,0)--(6,3.464101615137755)--(4,6.9282032302755105)--(0,6.928203230275511)--(-2,3.464101615137758)--cycle, linewidth(0.7)); 
 /* draw figures */
draw((0,0)--(4,0), linewidth(1)); 
draw((4,0)--(6,3.464101615137755), linewidth(1)); 
draw((6,3.464101615137755)--(4,6.9282032302755105), linewidth(1)); 
draw((4,6.9282032302755105)--(0,6.928203230275511), linewidth(1)); 
draw((0,6.928203230275511)--(-2,3.464101615137758), linewidth(1)); 
draw((-2,3.464101615137758)--(0,0), linewidth(1)); 
draw((-2,3.464101615137758)--(6,3.464101615137755), linewidth(1)); 
draw((4,0)--(4,6.9282032302755105), linewidth(1)); 
 /* dots and labels */
dot((0,0),dotstyle); 
dot((4,0),dotstyle); 
dot((6,3.464101615137755),dotstyle); 
dot((4,6.9282032302755105),dotstyle); 
dot((0,6.928203230275511),dotstyle); 
dot((-2,3.464101615137758),dotstyle); 
dot((4,3.464101615137756),linewidth(4pt) + dotstyle); 
clip((xmin,ymin)--(xmin,ymax)--(xmax,ymax)--(xmax,ymin)--cycle); 
filldraw((0,0)--(4,0)--(4,3.464101615137756)--(-2,3.464101615137758)--cycle,lightred,red);
filldraw((4,3.464101615137756)--(4,6.9282032302755105)--(6,3.464101615137755)--cycle,lightcyan,blue);

 /* end of picture */
    \end{asy}
\end{center}
\end{problem}
\vspace{0.3cm}
\begin{problem}
   A un tablero de $4\times 4$ se le retiran dos esquinas opuestas. ¿Puede cubrirse el tablero con 7 dóminos (rectángulos de $2\times 1$)?
    \begin{center}
    \begin{asy}
 /* Geogebra to Asymptote conversion, documentation at artofproblemsolving.com/Wiki go to User:Azjps/geogebra */
import graph; size(6cm); 
real labelscalefactor = 0.5; /* changes label-to-point distance */
pen dps = linewidth(0.7) + fontsize(10); defaultpen(dps); /* default pen style */ 
pen dotstyle = black; /* point style */ 
real xmin = -4.971088291553408, xmax = 14.090705728380154, ymin = -4.674638075103188, ymax = 6.902770562770564;  /* image dimensions */

 /* draw figures */
draw((0,0)--(0,3), linewidth(1)); 
draw((0,3)--(1,3), linewidth(1)); 
draw((1,3)--(1,4), linewidth(1)); 
draw((1,4)--(4,4), linewidth(1)); 
draw((4,4)--(4,1), linewidth(1)); 
draw((4,1)--(3,1), linewidth(1)); 
draw((3,1)--(3,0), linewidth(1)); 
draw((0,0)--(3,0), linewidth(1)); 
draw((1,0)--(1,3), linewidth(1)); 
draw((2,4)--(2,0), linewidth(1)); 
draw((3,0)--(3,4), linewidth(1)); 
draw((1,3)--(4,3), linewidth(1)); 
draw((0,2)--(4,2), linewidth(1)); 
draw((0,1)--(3,1), linewidth(1)); 
draw((6,3)--(6,1), linewidth(1)); 
draw((6,1)--(7,1), linewidth(1)); 
draw((7,1)--(7,3), linewidth(1)); 
draw((7,3)--(6,3), linewidth(1)); 
draw((6,2)--(7,2), linewidth(1)); 

 /* dots and labels */
clip((xmin,ymin)--(xmin,ymax)--(xmax,ymax)--(xmax,ymin)--cycle); 
 /* end of picture */
\end{asy}
\end{center}
\end{problem}
\vspace{0.3cm}
\begin{problem}
[\href{https://artofproblemsolving.com/community/c6h3137863}{Regional del Sureste Mexico 2023/1}]
Víctor escribe todos los números de $7$ dígitos usando los dígitos $1, 2, 3, 4, 5, 6,$ y $7$ exactamente una vez. Demuestre que no hay dos números tal que uno sea el doble del otro
    
\end{problem}
\vspace{0.3cm}
\begin{problem}
    Calcular el valor de 
    \[\frac{1}{\sqrt{1}+\sqrt{2}}+\frac{1}{\sqrt{2}+\sqrt{3}}+\frac{1}{\sqrt{3}+\sqrt{4}}+\cdots+\frac{1}{\sqrt{99}+\sqrt{100}}\]
\end{problem}

\end{document}