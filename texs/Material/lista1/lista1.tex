\documentclass[11pt]{scrartcl}
\usepackage[sexy]{evan}
\usepackage{amsthm}
\usepackage[helvetica]{quotchap} 

\title{hola}
\author{AlanLG}
\date{9 de Marzo del 2024}

\begin{document}

\maketitle

\epigraph{\textit{Augustin}: Comprendo el principio, pero ¿para qué sirve?

\textit{Évariste:} Para pensar de manera amplia. […] También para ir más allá. Permite anticiparse.

\textit{Augustin:} ¿Pero a qué se dedica?

\textit{Évariste:} No te entiendo…

\textit{Augustin:} Las aplicaciones concretas.

\textit{Évariste:} Ah eso… Ninguna.}
{\emph{Évariste Galois}}


\begin{problem}[\href{https://www.proyectomuro.com/OMMEB/2020/#Nivel2}{OMMEB 2020 N2}]
    
    Una máquina A es capaz de cortar una determinada área de césped en $120$ minutos. Una máquina B corta esa misma área en $240$ minutos. Si se ponen a trabajar las dos máquinas al mismo tiempo para cortar esa área, ¿en cuántos minutos estará cortado el césped?
    
\end{problem}

\begin{problem}
    En la siguiente figura, ¿Cuál es el área de cuadrado inscrito al triángulo rectángulo?
\begin{center}
\begin{asy}
    size(6cm);
    draw((0,0)--(30,0)--(0,15)--cycle);
    filldraw((0,0)--(10,0)--(10,10)--(0,10)--cycle, lightred , red);
    draw((-1,10)--(-1,15));
    draw((-0.7,10)--(-1.3,10));
    draw((-0.7,15)--(-1.3,15));
    draw((10,-0.7)--(10,-1.3));
    draw((30,-0.7)--(30,-1.3));
    draw((10,-1)--(30,-1));
    label("20",(20,-1),S);
    label("5",(-1,12.5),W);
    
\end{asy}
\end{center}
\end{problem}

\begin{problem}
[\href{https://artofproblemsolving.com/community/c5h2765743p24218156}{AMC 8/2022}]
La siguiente cuadrícula debe llenarse con números enteros de tal manera que la suma de los números en cada fila y la suma de los números en cada columna sean iguales. Faltan cuatro números. El número $x$ en la esquina inferior izquierda es mayor que los otros tres números que faltan. ¿Cuál es el valor más pequeño posible de $x$?
\begin{center}
    \begin{asy}
        unitsize(0.46cm);
draw((3,3)--(-3,3));
draw((3,1)--(-3,1));
draw((3,-3)--(-3,-3));
draw((3,-1)--(-3,-1));
draw((3,3)--(3,-3));
draw((1,3)--(1,-3));
draw((-3,3)--(-3,-3));
draw((-1,3)--(-1,-3));
label((-2,2),"$-2$");
label((0,2),"$9$");
label((2,2),"$5$");
label((2,0),"$-1$");
label((2,-2),"$8$");
label((-2,-2),"$x$");

    \end{asy}
\end{center}
    
\end{problem}
\begin{problem}
    Demuestra que el número
    \[\frac{1}{3}+\frac{1}{5}+\frac{1}{7}+\cdots+\frac{1}{101}\]
    No es un número entero
\end{problem}

\end{document}